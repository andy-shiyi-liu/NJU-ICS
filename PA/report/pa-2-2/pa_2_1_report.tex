%!TEX program = xelatex
\documentclass[UTF8]{ctexart}
\usepackage{ctex}

\usepackage{geometry}% 版面大小
\geometry{a4paper,scale=0.7}

\usepackage{graphicx}
\graphicspath{{assets/}}
\usepackage{caption2}
\usepackage{subfigure}
\usepackage{float}



\usepackage{listings}
\usepackage{xcolor}      %代码着色宏包
\usepackage{CJK}         %显示中文宏包

\usepackage[colorlinks,
            linkcolor=black,
            anchorcolor=blue,
            citecolor=green
            ]{hyperref}


\definecolor{mGreen}{rgb}{0,0.6,0}
\definecolor{mGray}{rgb}{0.5,0.5,0.5}
\definecolor{mPurple}{rgb}{0.58,0,0.82}
\definecolor{backgroundColour}{rgb}{0.95,0.95,0.92}

\lstdefinestyle{CStyle}{
    backgroundcolor=\color{backgroundColour},
    commentstyle=\color{mGreen},
    keywordstyle=\color{magenta},
    numberstyle=\tiny\color{mGray},
    stringstyle=\color{mPurple},
    basicstyle=\footnotesize,
    breakatwhitespace=false,
    breaklines=true,
    captionpos=b,
    keepspaces=true,
    numbers=left,
    numbersep=5pt,
    showspaces=false,
    showstringspaces=false,
    showtabs=false,
    tabsize=2,
    language=C
}

% \lstset{language=C}
% \begin{lstlisting}[style=CStyle]
% #include <stdio.h>
% int main(void)
% {
%    printf("Hello World!"); 
% }
% \end{lstlisting}

\title{\textbf{PA2-2 实验报告}}
\author{刘时宜 201180078}
\date{\today}

\begin{document}
\maketitle
\tableofcontents

\section{实现过程}
首先,修改\verb|testcase/Makefile|,更改程序装载地址。

然后,只需要补充\verb|kernel|中幅值内存内容的部分。由于是对内存进行操作,第一版想用\verb|vaddr_write|和\verb|vaddr_read|函数进行复制,但是这两个函数应当是\verb|NEMU|的内部功能,相当于硬件操作,\verb|kernel|不能够调用。于是改用指针方式,逐字节复制内存内容。

\lstset{language=C}
\begin{lstlisting}[style=CStyle]
// @ kernel/src/elf/elf.c
/* TODO: copy the segment from the ELF file to its proper memory area */
    for(addr_shift = 0; addr_shift < ph->p_filesz; addr_shift++)
    {
        // vaddr_write(ph->p_vaddr+addr_shift, 0, 1, vaddr_read(ph->p_offset+addr_shift, 0, 1));
        pdata = (void*)ph->p_offset+addr_shift;
        data = *pdata;
        pdata = (void*)ph->p_vaddr+addr_shift;
        *pdata = data;
    }
/* TODO: zeror the memory area [vaddr + file_sz, vaddr + mem_sz) */
    for(addr_shift = ph->p_filesz; addr_shift < ph->p_memsz; addr_shift++)
    {
        // vaddr_write(ph->p_vaddr+addr_shift, 0, 1, 0);
        pdata = (void*)ph->p_vaddr+addr_shift;
        *pdata = 0;
    }
\end{lstlisting}

以上代码中的\verb|pdata|、\verb|data|分别为\verb|char*|和\verb|char|类型,目的是进行逐字节的数据操作。

\verb|make test_pa-2-2|之后竟然十分丝滑地一次成功了。

\section{遇到的bug以及解决}
\subsection{没有修改Makefile导致的指令错误}
我采取代码下载下来,在本地编写程序并运行后再上传到服务器的方式完成PA作业。然而,应当是由于gcc版本不同的问题,本地的gcc编译时会产生 \verb|endbr32| 和 \verb|endbr64| 指令,均为英特尔公司后来加上的检查跳转位置是否合法的指令,在\verb|i386|的实现中均不存在,因此在 \verb|Makefile| 的 \verb|CFLAGS| 中加入了 \verb|-fcf-protection=branch -mmanual-endbr| 以取消这两条指令的自动生成。

然而,在PA服务器上的gcc却提示这两参数非法。故采取同步时不同步这两条指令的方法解决这个问题。

后果是,在本地修改的\verb|testcase/Makefile|并没有同步到服务器上,导致内存加载混乱,使得程序行为与反汇编结果不相符的奇怪情形。

在qq群中求助大家后顺利解决了这个问题。这里特别感谢陈泰霖同学告知使用 NEMU 的 monitor,使用\verb|si|命令加数字的方法使NEMU逐条打印执行的语句以及反汇编结果,使得我在混乱中最终定位了问题所在成功解决。

\subsection{本地运行结果与服务器运行结果不一致}
\verb|kernel|写好后,在本地运行会出现程序死循环的情况,而在PA服务器上运行则可以通过所有\verb|testcase|。首先通过反汇编结果发现PA服务器与本地编译的\verb|kernel|并不一致。然后通过使用参考指令实现(\verb|__ref_|)的方式发现本地可以正常运行,说明问题在于NEMU的指令实现。最终确定有问题的指令为\verb|push_i_b|。

检查代码后发现问题居然是源操作数的\verb|data_size|没有指定为8。修改后在服务器和本地均能成功运行。

如此低级的错误竟然能够通过所有\verb|pa2-1|以及服务器上\verb|pa2-2|的测试,说明测试样例确实不够完善。或许可以借鉴\verb|kernel|的程序行为开发一些新的能够检测这个错误的测试样例。暂且不在此叙述,构建好后再行单独提交测试样例。

\section{思考题}
\paragraph{为什么在装载时要将内存中剩余的 memsz-filesz 字节的内容清零?}~
\par 这部分内容为声明了但是没有初始化的全局变量的存储空间。将这部分内容清零,应当是 NEMU 约定所有全局变量的值均初始化为0。


\end{document}

% \begin{lstlisting}[style=CStyle]
% 
% \end{lstlisting}